% !TeX encoding = UTF-8
% !TeX spellcheck = en_GB
\section{Security}
\subsection{Risk Identification}
\subsubsection{Assets}
% Identify assets (e.g. web application)
\paragraph{} We have different assets that need to be protected:
\begin{itemize}
	\item The web application.
	\item The API.
	\item The database.
	\item Development, debugging, and deployment tools.
	      % \begin{itemize}
	      %  \item SSH.
	      %  \item Graphana/Prometheus.
	      %  \item GitHub.
	      %  \item GitHub Actions.
	      %  \item GitHub Secrets.
	      %  \item DigitalOcean.
	      % \end{itemize}
\end{itemize}

\subsubsection{Threat sources}
% Identify threat sources (e.g. SQL injection)
\paragraph{} Thread sources can be diverse, and acttack different layers of our application:
\begin{itemize}
	\item Vulnerabilities in our code: Among the vulnerabilities that we \textbf{could} have, these are the most relevant ones:
	      \begin{itemize}
		      \item SQL injection.
		      \item Cross Site Scripting (XSS).
	      \end{itemize}
	      We need to emphasise that we have checked our code for such vulnerabilities, and we haven't found any.
	\item Missing HTTPS and firewall.
	\item Misconfigured firewall.
	      % \item Stolen SSH keys for the VM.
	      % \item Stolen credentials for Graphana.
	\item Public Prometheus metrics.
	      % \item Stolen credentials for GitHub.
	      % \item Stolen credentials for DigitalOcean.
	\item Stolen credentials for all dependencies that require authentication
	\item Vulnerabilities in the OS or Docker Images used.
	\item Supply chain attacks.
\end{itemize}

\subsubsection{Risk scenarios}
% Construct risk scenarios (e.g. Attacker performs SQL injection on web application to download sensitive user data)
\paragraph{} Some possible scenarios of attacks are as follows:
\begin{enumerate}
	% Note: Add only elements at the end of the list, or you'll have to modify the risk matrix below!
	\item SQL injection. %to download sensitive user data.
	\item XSS on web application.% to download sensitive user data or deliver malware.
	\item Snooping on user traffic.% and obtains a user's password, which then uses to impersonate them.
	\item man-in-the-middle attack.% on a user and provides them with false information.
	\item Stealing of SSH keys.%, connects to the VM, from there to the DB, and downloads sensitive user data or deliver malware.
	\item Attacker steals Graphana credentials.%, connects to it, and learns more information about the usage of our system.
	\item Attacker connects to \textit{/metrics} and can surveil the system.
	\item Attacker steals GitHub credentials.% and modifies our code.% or CI/CD to introduce vulnerabilities, or execute arbitrary code.
	\item Attacker steals DigitalOcean credentials.% and downloads sensitive user data from the DB, or use our VM to deliver malware.
	\item Attacker steals DB credentials.% and connects directly to it, due to a firewall misconfiguration.
	\item Attacker uses vulnerabilities of our the OS or Docker images to connect to the DB.% and download user data, or to the VM and deliver malware.
\end{enumerate}

\subsection{Risk Analysis}
\subsubsection{Risk Matrix}
% Determine likelihood
% Determine impact
% Use a Risk Matrix to prioritize risk of scenarios
The risk matrix below helps visualize the possible risks in terms of impact and probability. %We classified the above scenarios according to their likehood and probability on the following matrix. Note that the numbers on the matrix correspond to the numbers on the list of risk scenarions above. \\

\begin{tabular}{|l|l|l|l|}
	\hline
	                           & \textbf{High Impact} & \textbf{Medium Impact} & \textbf{Low Impact} \\ \hline
	\textbf{Low Likelihood}    & 5,8,9,10             & -                      & -                   \\ \hline
	\textbf{Medium Likelihood} & 11                   & 1,2                    & 6                   \\ \hline
	\textbf{High Likelihood}   & -                    & 3,4                    & 7                   \\ \hline
\end{tabular}

\subsubsection{Steps to take}
% Discuss what are you going to do about each of the scenarios
\paragraph{} In order to minimize our risk, we are going to take the following steps:
\begin{itemize}
	\item Encrypt our SSH key with a passphrase.% in order to make it more difficult to steal.
	\item Hide Graphana behind a firewall.
	\item Limit the access to \textit{/metrics}.
	\item Use two-factor-authentication in all GitHub and DigitalOcean accounts.% of the team members.
	\item Continue using best practices when developing, regarding SQL injection and XSS attacks.
	\item Be aware of the dependencies we are using and their potential vulnerabilities, and keep them updated.
\end{itemize}

\subsection{Steps taken}
% Based on your security assessment, take steps to improve the security of your system. Briefly report on the steps you took.

\begin{itemize}
	\item We encrypted our SSH key with a strong passphrase.%, which teams members keep separate from the key. 
	\item Most GitHub and DigitalOcean accounts already had 2FA enabled, and we enabled it in all the remaining ones.
	\item We decided against hiding Grafana behind a firewall since it needs to be access by third parties, namely the professors.
\end{itemize}

