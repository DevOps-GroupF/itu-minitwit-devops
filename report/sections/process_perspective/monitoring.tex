% !TeX encoding = UTF-8
% !TeX spellcheck = en_GB
\section{Monitoring}
\subsection{Prometheus}
\subsubsection{Decision Log}

We chose Prometheus as it is a is a powerful and widely used open-source tool for systems monitoring.
%We chose this tool for its open-souce nature, ease of use, powerful query language and simple interoperability with Grafana.
%Prometheus was used as a docker container running on our VMs and it was used for quering monitoring data about our running .NET minitwit application
Setting up Prometheus really was as easy as just getting a docker image of it running.
All the data source configuring was done through Grafana.%, including all visualization of the data itself. This ease of use was very evident and convenient.

\subsection{Grafana}

\subsubsection{Decision Log}

Grafana is an open-souce tool, that we decided to use for its ability to visualize all our required metrics. %In addition, multiple data sources can be easily set up for a single visualization dashboard, making the monitoring of our application easy.

%Together, Prometheus and Grafana form a comprehensive monitoring solution that addresses both data collection and visualization needs. Prometheus's robust data collection, combined with Grafana's powerful visualization and dashboarding features, provide a seamless and efficient way to monitor and manage all our systems.

\subsection{Implementation of monitoring}

A docker container of Prometheus and Grafana is running on the manager node in our docker swarm.
Prometheus has been programmed through Grafana to gather and display metrics on all MiniTwit instances. % in all worker nodes and display them on the manager nodes Grafana dashboard.
%Metrics would be gathered on \{url\}:8080/metrics and that is where Prometheus can find them.
In addition, Grafana gathers data from a managed DigitalOcean database, and displays metrics such as total followers and total tweets.
Figure \ref{fig:Monitoring2} describes how monitoring was implemented.

\begin{figure}[H]
	\centering
	\includegraphics[width=0.6\textwidth]{Monitoring2.png}
	\caption{Diagram describing our monitoring implementation}
	\label{fig:Monitoring2}
\end{figure}
\begin{figure}[H]
	\centering
	\includegraphics[width=1\textwidth]{metrics.png}
	\caption{Example of a worker nodes metrics page}
	\label{fig:metrics}
\end{figure}

\subsection{Monitoring configuration}

Grafana and Prometheus are configured in a way, so that when initiating a docker swarm with a manager nodes running docker images of them, then all necessary configuration gets set up automatically.

\subsubsection{Prometheus configuration}

Prometheus gets configured using a YAML file that gets inserted into the manager nodes Prometheus Docker image as a volume through a YAML file.

\subsubsection{Grafana configuration}

Grafana data sources are also set up using a YAML file that gets inserted into the Grafana docker image of the manager node, by using volumes. This file also
%The fon 
contains variables about how to set up a Postgres database connection and a Prometheus monitoring connection.

Additionally, two dashboards get added to the Grafana Docker image as json files, which contain the configuration of them.
%They contain all the variables and setup parameters required to automatically set up the required dashboards.

Users are set up using a shell script:
After the Grafana Docker image is running, a shell script sends a curl API call to the running Docker image every 5 seconds until success, in order to add the neccesary users.
%The API call contains the users that are needed to set up automatically.
